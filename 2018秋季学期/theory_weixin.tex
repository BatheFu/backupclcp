\section{微信健康传播}
健康传播的概念发轫于上世纪70年代斯坦福大学的“斯坦福心脏预防计划”,旨在“通过减轻体重、减少吸烟、降低血压血脂水平来降低心脏病发病风险”。其发展和成熟阶段在20世纪的80、90年代,诸多相关研究在美国数十所大学开展,并在80年代发行了奠基性的著作《健康传播:理论与实践》,1996年发行代表刊物《健康传播研究》。

我国的健康传播在80年代后得以确立,尽管各地有建立零星的机构和团体,但是其实际影响力较小,处在学界研究的边缘,这使得真正有组织、有计划的研究在90年代后才得以开展。由于相关论著的缺乏和实证研究方法上的不成熟,我国的健康传播研究相对其他发达国家拉开了较大的差距。

目前为止,健康传播在全球的研究领域通常包括九个方向的研究: 1、大众健康传播媒介和效果研究;2、健康传播组织研究;3、以“医患关系”为轴心的人际健康传播研究;4、健康教育与健康促进不同的研究;5、健康传播的外部环境的研究;6、健康传播与文化的研究;7、艾滋病、安乐死、同性恋、器官移植等特殊议题的研究;8、健康传播史的研究;9、突发公共卫生事件”。\cite{张自力2005}

其中大众传播媒介和效果研究是一大研究热点。当前处于互联网整合进生活的数字时代,健康传播依赖的载体不断变化,传播行为形式多样,参与主体日益丰富,也显示出多元的模式特征。因而“互联网+医疗”应运而生并且日益受到学者关注,根据中国知网的指数分析,从2014年开始,相关公开发表的研究呈现指数级急剧增加,近两年年发文量超过500篇。

2018年4月28日,国务院发布《国务院办公厅关于促进“互联网+医疗健康”发展的意见》,其中第三项第六条提出,建立网络科普平台,利用互联网提供健康科普知识精准教育,普及健康生活方式,提高居民自我健康管理能力和健康素养。\cite{互联网+}

这体现出健康传播和互联网紧密结合的态势和积极的政策导向,为此研究互联网健康传播的主体、内容、策略、效度、局限性和当前问题显得具有很好的现实效应。下面以互联网健康传播的主要阵地之一,微信公众平台做一探讨。
\subsection{微信健康传播}
\subsubsection{背景}
微信公众平台在 2012 年 8 月正式推出,可以实现包括文字、图片、语音等多种形式的互动。近年来,微信平台一直保持活跃态势,用户不断增长,公众号数量逐年增加。根据《2017年微信经济数据报告》,截至2017年底微信公众号已超过1000万个,其中近六成保持活跃,其中健康类公众号占据半数江山。月活跃粉丝数为 7.97 亿,同比增长 19\%,公众号已成为用户在微信平台上使用的主要功能之一。

随着物质生活水平的提升,日常生活的精细化,健康养生信息成为民众关注的热点,使得养生信息具备了特殊性\cite{高会芳2017}。对于微信平台本身,这同样顺应了其商业化的需求,也是对社会健康风险发力的好机会,响应了健康中国建设的时代潮流。

\subsubsection{主体}
微信公众平台的健康传播主体大抵分为企业、个人、医院机构等大类。
范树璇\cite{樊树璇2016健康类微信公众号传播力及优化策略研究}将健康类公众号分为纸媒类、互联网类、医院机构类、草根类四类。对于主体各自的重要性,
刘艳丽\cite{刘艳丽2016健康类微信公众号主体分析}认为微信公众号主体中企业占半数,是不可忽视的一个健康传播主体。但是就阅读量来看,根据范树璇的研究,互联网媒体和民间公众号最高,其次为传统纸媒和医院机构,前者的主体性更受到更多关注。
\subsubsection{策略和内容}
标题是决定受众是否点进链接看推送文章的决定性因素,因而运营者重视在标题书写的策略有“标题长”、“亲和力”、“句式多样化”的特点,制作策略有“制造信息阶梯”、“列数字”等\cite{刘婷2018健康类微信公众号文章标题的制作规律——基于“生命时报”“丁香医生”等公众号的分析}特点。同样,微信提供一套原创性标签以对原创内容进行保护。就内容原创性而言,媒体>医院机构>民间公众号,内容信息主要为医疗信息、医院新闻以及泛文化养生信息。内容满意度上互联网媒体最高,民间公众号最低。\cite{樊树璇2016健康类微信公众号传播力及优化策略研究}传播内容考虑到受众的阅读时间和状态,挑选的内容大多易读。
金晓玲等\cite{金晓玲2017微信朋友圈中健康信息传播行为研究}提出,对电子健康信息的传播行为影响最大的因素为富含情绪性、有用性和有趣性。但是在这一选择也使得精品难见,文章用语疲乏,抄袭现象严重等情况。

对于已经建立起较大规模粉丝群的公众号而言,其传播策略较为稳定合理,方便总结出一套科学的可借鉴的范式。
农海燕\cite{农海燕2016论微信健康传播的“知信行”范式}等研究了丁香医生公众号的传播策略,发现1.在就餐时间推送2.贴近新闻热点3.形式图文结合,传播内容以身体保健(34\%)和疾病科普(23\%)为主。
张雯\cite{张雯2018}研究了营养科学专家范志红的个人微信公众订阅号“范志红原创营养信息”,提出灵活使用恐惧诉求、“晓之以理,动之以情”、明示结论为主和给出可操作性强的建议这四种说服传播技巧。
\subsubsection{效度和评价}
尽管部分研究肯定了微信公众平台对健康传播的正向作用,但是由于信息碎末化、飞沫化的特点,健康传播一旦内容谬误百出,那么非但不能起到科普作用,反而会引出不少负面效果。有学者认为,微信平台健康养生信息的泛滥与健康知识的传播活动相去甚远,与其说健康传播在微信平台的兴起,毋宁说是包装为健康信息的内容营销方式的兴起。\cite{李东晓2016}

同时,微信传播的热度中心迁移迅速,寻得长久的增长模式不容易。
方靖\cite{方婧2016微信公众号信息传播热度的影响因素实证研究}等发现传播热度和主题、时间、特征有一定相关性,和发送推文的频率相关度低,热点增长模式为激增模式。这就使得长久的经营容易被短时激增挫败,让经营者难以得到传统的认同感。
丹娜·巴吾尔江\cite{丹娜·巴吾尔江2018健康类微信公众号的传播能力研究}在清博大数据的平台基础上,考察了微信传播主体、对象、内容,建立了一套评价传播影响的权重体系。同时,他还提出非医学类企业公众号污染健康传播环境的现象,提醒我们需要为这一环境做出净化措施。
\subsection{利用微信平台进行干预}
目前国内已经有一定规模的微信干预研究,在医院、学校等健康意识渗透较高的地点开展了相关研究。涉及内容有大学生健康教育\cite{张华2017大学生健康素养微信公众平台干预效果评价}\cite{杨洋2017}\cite{张冬2016}、慢性病管理\cite{汤倩茜2016}、术后恢复\cite{黄邵薇2016}\cite{宋宏2015}、传染病科普\cite{蒋艳2017}等等。

这些研究大多遵循传统的实验室传统,将实验室环境迁移到数字环境当中,既节约了成本,也提高了获取信息的效率。遗憾的是,少有时间序列信息完整的研究,没能揭示受试者变化的动态过程。
\subsection{治未病微信公众号建设}
\subsubsection{受众和内容}

受众:针对青年群体和中年群体两方面的人群分别推出中医科普系列和养生传播系列文章。该人群有基本的读写和认知能力,其中青年群体的教育水平较高。

内容:中医科普系列包括:基本概念系列、常见中成药用法普及。
为了提供趣味性和可读性,还有一部分章节介绍古人的
世界观和日常起居,例如天气,星宿、方位、饮食、节气、农事,婚丧嫁娶等。

养生传播系列包括:运动保健、穴位保健、疾病科普、饮食保健、节气养生,从出版的书籍采编内容,保证内容的合理性和易读性。

\subsubsection{采用的策略}

\begin{itemize}
    \item 标题中采用信息阶梯
    \item 标题中添加数字
    \item 内容贴近生活,关注时事热点
    \item 文中图文结合,文末投票互动
    \item *每周测验小游戏
\end{itemize}

