\section{微信健康传播}
\subsection{文献回顾}

根据《2017年微信经济数据报告》,截至2017年底微信公众号已超过1000万个,其中近六成保持活跃。

范树璇\cite{樊树璇2016健康类微信公众号传播力及优化策略研究}将健康类公众号分为纸媒类、互联网类、医院机构类、草根类四类,并且每类选取3个,共15个代表样本对其主体实力、传播内容、传播效果进行了考察。

张华\cite{张华2017大学生健康素养微信公众平台干预效果评价}等通过微信公众平台对两所院校463名学生进行微信健康素养问卷调查和干预。

丹娜·巴吾尔江\cite{丹娜·巴吾尔江2018健康类微信公众号的传播能力研究}在清博大数据的平台基础上,考察了微信传播主体、对象、内容,建立了一套评价传播影响的权重体系。提出非医学类企业公众号污染健康传播环境。

李东晓认为,微信平台健康养生信息的泛滥与健康知识的传播活动相去甚远,与其说健康传播在微信平台的兴起,毋宁说是包装为健康信息的内容营销方式的兴起

刘婷\cite{刘婷2018健康类微信公众号文章标题的制作规律——基于“生命时报”“丁香医生”等公众号的分析}分析了微信公众号的制作规律,归纳出标题有“标题长”、“亲和力”、“句式多样化”的特点,制作策略有“制造信息阶梯”、“列数字”等。

刘艳丽\cite{刘艳丽2016健康类微信公众号主体分析}认为微信公众号主体中企业占半数,是不可忽视的一个健康传播主体。

方靖\cite{方婧2016微信公众号信息传播热度的影响因素实证研究}等发现传播热度和主题、时间、特征有一定相关性,和发送推文的频率相关度低,热点增长模式为激增模式。

范树璇\cite{樊树璇2016健康类微信公众号传播力及优化策略研究}考察了15家健康类微信公众号为期3个月的推送内容,对内容来源、内容信息和阅读量进行分析后得出以下结论:内容原创性媒体>医院机构>民间公众号,内容信息主要为医疗信息、医院新闻以及泛文化养生信息。阅读量互联网媒体和民间公众号最高,其次为传统纸媒和医院机构。内容满意度上互联网媒体最高,民间公众号最低。

金晓玲等\cite{金晓玲2017微信朋友圈中健康信息传播行为研究}提出,对电子健康信息的传播行为影响最大的因素为富含情绪性、有用性和有趣性。

农海燕\cite{农海燕2016论微信健康传播的“知信行”范式}等研究了丁香医生公众号的传播策略,发现1.在就餐时间推送2.贴近新闻热点3.形式图文结合,传播内容以身体保健(34\%)和疾病科普(23\%)为主。

\subsection{治未病微信公众号建设}
\subsubsection{受众和内容}

受众:针对青年群体和中年群体两方面的人群分别推出中医科普系列和养生传播系列文章。该人群有基本的读写和认知能力,其中青年群体的教育水平较高。

内容:中医科普系列包括:基本概念系列、常见中成药用法普及。
为了提供趣味性和可读性,还有一部分章节介绍古人的
世界观和日常起居,例如天气,星宿、方位、饮食、节气、农事,婚丧嫁娶等。

养生传播系列包括:运动保健、穴位保健、疾病科普、饮食保健、节气养生,从出版的书籍采编内容,保证内容的合理性和易读性。

\subsubsection{采用的策略}

\begin{itemize}
    \item 标题中采用信息阶梯
    \item 标题中添加数字
    \item 内容贴近生活,关注时事热点
    \item 文中图文结合,文末投票互动
    \item *每周测验小游戏
\end{itemize}

