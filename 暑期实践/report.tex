\documentclass{article}
\usepackage{ctex}
\usepackage{hyperref}
\usepackage{setspace}
\usepackage[left=3cm,right=3cm]{geometry}
\author{付裕刚}
\title{治未病暑期实践报告}
\date{\today}
\begin{document}
   \zihao{-2} \maketitle

    \begin{abstract}
        \zihao{-4}
        暑假通过一个月的时间试水这一课题得出以下几个结论:1.人群对于治未病意识较为薄弱,和经济水平和观念意识直接相关。2.治未病作为具体的实践项目在不同的机构扮演不同的角色,在社区卫生服务中心大多以治疗项目出现,在居委会成为零散的宣传项目。另外,本文中的社区医院和社区卫生服务中心一词同义。
    \end{abstract}
    \clearpage
    \zihao{-4}
    \section{挑战杯作品特点}
    就整个挑战杯项目而言,我们在挑战杯官方网站的作品库-社会分类下找到往期的作品进行参考。可以发现,近几年的作品要求明显较以往提高,各个获奖作品对于所研究的内容有具体的机制互动设计和阐释,和09年左右泛泛的广泛调查研究区别较大。
    
    同样是南京市的社区研究,来源自第13届挑战杯的作品提出构建“医院-社区-患者-志愿者”四位一体慢性病防治管理新模式。详见下面绿色标出的链接:\href{http://2017.tiaozhanbei.net/d37/project/124/}{慢性病模式构建}。
    
    该项目同时使用了定性和定量两种方法进行研究,定性主要指访谈法,定量主要指问卷法。另外,我预计该团队使用了相关的生存质量量表对于患者的健康状况进行了评估以纠正模式的运行。同时他们吸收了预防医学中“同伴支持”与“志愿服务”两个概念,旨在建立自发性的互助小组和他人干预性的志愿服务相互结合,管理慢性病的发展进程。
    \section{本项目思路}
    对于这一项目,值得借鉴的一点是体系化的思路,可以把平台建立的很完善。通过模式使医院、社区、患者、第三方等利益相关者密切的联系起来,也就可以很方便的把慢性病管理当前现状的优缺点暴露出来并且修正他,让各方之间的协作更加契合。另一点需要注意的是,挑战杯由于需要和相关政策密切结合,因此本着“健康中国2030”的大背景,再细细的梳理“治未病”相关的政策,可以看出预防是这一理念的核心,强调的更多的一种是主动:社区主动宣传,患者主动参与,医院主动服务,继而改变以往“有病才医”的求助式被动局面。
    
    思路一:延续使用上述的体系化思路。我们的项目要厘清居民的需求,明晰社区卫生服务中心所能提供的服务范畴和潜力,继而构建一种方式让居民诉求,居委会协调和社区医院的服务形成良性的互动。这样提高三者之间的内在亲和性,提高社区和居民健康之间的联动性。
    
   思路二:使用“冲突”的理论思路。试图去阐述居民和社区医院之间的内在冲突,也就是给出社区医院服务资源浪费和居民健康诉求尚未被满足的原因。继而看“治未病”是否有其特点来缓和这样的冲突。这样的冲突可以是信任/观念/制度问题、也可以是经济成本问题,治未病理论能否扭转观念、降低成本就是值得探讨的问题。
   
   思路三:通过构建微信平台向各个民营中医馆推送服务,包括治未病知识宣传,治未病医师“连线”,治未病实践指导,如针灸贴敷、膏方服务等。
    
    可以看出,无论是采用哪一种思路,都需要先运用访谈法对全体居民进行分层抽样访谈。
    \subsection{访谈法}
    采用半开放式访谈,即设计出一定数量的问题,但是在访谈过程中可以增加访谈范围。
    
    访谈对象:社区的居委会主任,社区医院相关负责人。
    
    操作细节:社区和社区医院之间具体分工和职责的联系可以在访谈之中相互映照,得出结论,不必直接提问以免对方出于谨慎和担忧不愿回答某些问题。对于社区医院的访谈,可以以负责人为线索,再向各个承担治未病服务的科室负责人访谈。
    \subsection{问卷法}
    发放先前设计好的知信行调查问卷,该问卷题量为30题左右,目的是测知居民对于治未病的基本了解和做法。鉴于有一定的题干阅读量,我们建议参与人员年龄不超过65岁以适应其要求的一定知识水平。
    
   先进行抽样,而后在明确住户楼层分布后进行问卷调查是较为理想的做法。但是,在实际操作中,此举被拒可能性大,成本高,因此需要转变成本较低的方法。最后酌定用互联网调查问卷以获得更好的效果。
    
    \subsection{知信行问卷思路和目的}
    具体见先前的主文件,简单说是对于知识/信念/行为三方面的简单评估。
    
    需要注意的是,知信行问卷并未涉及到居民和社区、社区医院之间的互动提问。例如居民去社区医院的频率、对社区医院的满意程度、居民对于社区宣传/组织工作的反馈等等。因为项目成本的问题,另外这和社区医院问卷必定有人群上的重合,因而预计这一信息的获取由居民访谈完成。
        \subsection{卫生服务中心问卷思路和目的}
        对于患者,着重想要了解他们来社区医院的原因、频率、服务项目、建议、对于其他组织参与的认同度等等。
        
        *对于医院,则了解医院的基本情况,以及他们对于治未病的看法和当前建设的困难所在。这一工作设计的人群小,由访谈完成。
            \subsection{居民访谈思路和目的}
            对于不同年龄层次的居民进行访谈,主要厘清他们对于治未病的印象、期望、生活习惯、投入成本估计、需要解决的健康问题等等
             \subsection{社区访谈思路和目的}
            从居委会了解到居民的来源和年龄分布,以及平日工作中涉及到居民预防保健的部分。我们认为居委会是连结居民和社区医院、第三方服务的纽带之一,因此可以询问居委会是否和卫生服务中心有联动,以及是否知晓第三方志愿者团体的参与等等。
    \section{计划变更}
        经过几次实地问卷发放,我们发现暑期问卷发放所需人力不够。入户问卷调查的成功率小于成,且由于很多居民白天时段需要上班,上门的时间非常有限。而在夜晚的私人时间,大多数人也不愿意被其他事情打扰。入户问卷3组,耗时3小时不过得到实际有效问卷10份,效率非常低下。
        
        而户外的即时性的问卷发放,先有一小区进行,在人群密集的时间段和地点进行统一服装的问卷发放。另务必登记楼层,以减少同一楼的出现频率过高,造成过大的偏差。这样做的弊端依旧是老年人居多,大多数不愿意填写如此题量的问卷,且有相当一部分人不识字,同样无法完成。因而在暑期改由访谈为主,先进行定性考察。
        
        如是,我们迫于形势将重心改为访谈,而问卷填写经与居委会协商后,或在九、十月社区举行活动时随活动一同发放,或学校相关社团在进行活动时帮忙分发,但是人群的分布仍就无法合理的掌握。
        
        访谈方面,暑期期间访谈了各区划中的社区卫生服务中心和社区居委会相关负责人,并从中获得一些和“治未病”相关的信息。
    
    在八月,我们打算查阅相关的文献,明晰下一步的思路,并且根据该比赛的特点,需要把握当前政策的导向,依托新的提法和关键词来找到更好的创新点。
    
    开学后继续招募成员,分组进行走访和问卷的完成,以期望找出现有模式的缺陷,分析未病政策实施困难的原因和机制分析,分析各方参与扮演的角色和互动,并提出新的具有可行性的某类措施。
    \section{结果分析}
    \subsection{居民访谈}
    从居民访谈来看,居民遇病才会求医,不会主动寻求保健服务,对于保健知识中青年也没有过多的追逐热情。对于老年居民,他们有切身的健康诉求,但是往往由于社区医院的服务不够完善而流向大型三甲医院。反过来说,由于社区医院资源有限,居民难免会有轻侮的心理,不信任社区医院的服务,导致基层医院对于中青年人的吸引力很小。对于职工待遇好的老年人,他们则直接和相应的定点医院联动,也不和社区医院有所交集。
   
    \subsection{社区医院访谈}
    从目前的社区医院访谈来看,来访人群以老年人和小孩为多。他们寻求的服务主要为治病,其次为职工的定期体检,来进行针对疾病先兆期的治疗的较少,且这一现象仅仅出现在中医科室。社区医院服务的人群基数为五万到六万人,覆盖5-8个社区,接诊量在每日200人以上,但是一般不超过500人。
    
    其次,社区医院的财政开支由区政府负责,社区医院根据自身的营收情况会对服务的项目进行调整。各个区之间的拨款相差额度可能较大,但是区域内的拨款大致相同。社区医院根据实际情况开设诊室,提供相关服务,但是医院不可能开设未被批准的项目,即服务项目必须在某文件中明确出现。因此,社区医院基于他的性质,不可能像民营医院一样自由地进行医疗项目试点。
    
    在我们走访的社区医院,均未开设治未病科室,仅有两家提供了中医体质辨识服务。该服务纳入体检项目当中,其中65周岁以上的纳入健康档案,医生根据体制量表获得的结果提供相应的生活方式上的建议,并且在之后进行随访。以大光路社区医院为例:
    
    大光路社区卫生服务中心开展了中医体质辨识服务,目前该项目附在体检项目之后,未予额外的收费,作为类似于试点项目的存在。该社区居民文化层次较高,有一定的预防保健想法,因此会接受体质辩证的服务,尽管其中有很大的一部分是体检项目附带的原因。该项目刚开始进行试点两个月,在我们同负责人庄医生的访谈后,得知此项目除体检来源外,有一部分年轻群体进行咨询,且防病需求在他看来是普遍存在,防病意识在逐步地提高。
    
    在经济负担方面,社区医院收费标准较低,且项目大多纳入了医保范围。在蓝旗社区卫生服务中心,我们获知,前来社区医院就诊的居民大多为企业退休职工,不具有较好的预防保健意识。另外,对于退休收入更低的人群而言,倘若他们患有慢性病,则凭借医保得以勉强负担降压药和胰岛素等需长期服用的药物,更不可能自费进行一些未病咨询或服务。
    
    如果居民对于社区医院的信任有限,社区医院即使是提供附带式的免费服务,也没有办法真正体现治未病的价值,让这一服务体系形成良性互动。
    
    \subsection{居委会访谈}
    就联系居民的方式而言,居委会主要通过网格制度和居民取得联系。网格一般指社区按居民楼分布进行的区域性划分。一个社区工作人员分管5-6栋居民楼,规模500人左右,形成了一个网格。居委会利用网格制度,一方面向居民传递信息,另一方面接受居民对社区建设的意见。同时,东方城社区充分利用互联网,建立了QQ、微信群等;五福家园社区通过居民骨干,也就是楼层长制度来充当联系人的角色,同样类似网格所代表的区域划分思路。
    
    此外,居委会组织一定程度的预防保健活动,他们基于公共卫生管理的考量,其内容有一部分属于中医治未病的范畴。例如,东方城居委会通过立项的方式响应居民需求。在预防保健方面,开设健康讲堂、定期组织职工体检、关心走访慢性病人、同社区医院联动进行义诊等等。可以发现,尽管上述的服务涉及到中医治未病的部分,但是尚未出现专门有以治未病为名义的相关立项。不管是居民还是居委会,他们都表现出对于治未病一词的陌生,但是如果讲这一词语的内涵缩小至“预防疾病”这一种,那么就能得到理解。
   \clearpage
    \section{剩余工作}
    \subsection{产生的问题}
    梳理一下尚没有搞清楚的问题,主要有以下的几个:
    \begin{itemize}
        \item 蓝旗社区医院和大光路社区医院相隔不远,但是周围居民预防保健意识差别很大,由是可以从下一部分的思路二进一步探究机制和原因。即从居民的收入、教育背景等可以量化的角度进行分析,并且看这些因素是否可以促进社区医院的再生产和再投入。
        \item 治未病理论的防病/治疗疾病先兆/病后防变/预防复发这几个不同含义分别可以对应哪些服务,其中又有哪些可以进入社区医院的体系,哪一些可以由第三方完成?目前具有现实可行性的服务又是哪些?
        \item 有能力接受自主治未病服务的收入界限是什么?
   \end{itemize}
\subsection{人员具体安排}
陈薇羽:新闻稿和政策梳理

付裕刚:走访和综合写作

周宇:走访和文字稿整理

葛任洁:机构联系和文献搜集

汤珂安*:统计方案设计和分析

拟招:走访工作*2;计算机平台搭建*1;经济、营销相关*1

    \section{相关问卷和访谈设计}

    \subsection{医院访谈设计}
1.您这里有治未病服务吗?具体有哪些?

2.成立了专门的科室吗?还是分散于各个科室?

3.相关科室需要的资金、设备投入?

4.每天的服务量有多少,离得远的社区居民也会来吗?

5.患者经济能力?奔着什么项目来?自费还是医保?

6.响应了哪些相关的治未病政策?

7.服务中心有进行治未病宣传/保健宣传吗?

8.有相关知识宣传、义诊等活动吗?

9.有针对特殊群体,比如针对老年人的服务吗?

11.社区对于治未病有什么规划和期望吗?

13.患者会转去上级医院吗,大致有多少?

14.可访问的相关数据和相应的授权是什么?

目前社区卫生服务的具体实践措施有建立健康档案、妇幼保健、老年人管理、慢性病患者,但是针对年轻群体和健康群体的较少。


    \subsection{居委会访谈设计}
1.居委会联系居民的方式?渠道?频率?

2.居委会在预防保健方面的工作?

3.对于特殊群特的关照?

4.第三方志愿者的组织和活动情况?

5.信息网络化?
    \begin{spacing}{0.9}
            \section{问卷小样}
您好,这是一份由南京中医药大学第一临床医学院发起的调查问卷,旨在了解社区居民对于中医治未病理论的认知、行为等情况,响应“健康中国”工程,为后面中医治未病理论的传播做一个描述性统计分析。

本问卷时长约五分钟,我们郑重承诺,本问卷的所有数据仅用于科研用途,您的所有个人信息将受到相关法律的保护。

\begin{enumerate}

\item
偏方是中医的一种。\\
A.正确\\
B.部分正确\\
C.错误

\item
煎药时间越久越好。\\
A.正确\\
B.部分正确\\
C.错误

\item
食疗是中医的治疗手段\\
A.正确\\
B.部分正确\\
C.错误

\item 
治未病是中医特有的理论和方法
A.正确\\
B.部分正确\\
C.错误

\item 中医的存亡取决于中医是否有效
A.正确\\
B.部分正确\\
C.错误

\item
过去的一年里,您接受过中医治疗吗?(Hon第一题)\\
A. 有\\
B. 没有

\item
过去一年您接受中医治疗的次数?\\
A.1-2次\\
B.2-5次\\
C.5次以上

\item
您接受了哪一种中医治疗呢?\\
A.草药煎汤\\
B.贴敷的膏药\\
C.物理疗法,比如针灸,拔火罐等\\
D.填写

\item 
您参加过养生知识的讲座吗?
A. 有\\
B. 没有

\item 
未来您会去接受中医治疗吗?\\
A.会\\
B.不会\\
C.可能


\item 您喜欢看养生方面的书籍、视频等资料吗?
A.喜欢\\
B.一般\\
C.不喜欢\\

\item 
您对于当前国家鼓励中医药发展怎么看?\\
A.我支持\\
B.我保持中立\\
C.我反对

\item 
我们知道,电视台会播放中医节目,也有微信文章进行中医知识的传播。
\begin{enumerate}
	\item 
   您收听/观看此类节目吗?\\
    A.从不\\
    B.偶尔\\
    C.经常

\item 
    您阅读此类文章吗?\\
    A.从不\\
    B.偶尔\\
    C.经常
    
    \item 
    您喜欢这些节目/文章吗?\\
    A.喜欢\\
    B.一般\\
    C.不喜欢
    
    \item 
    您会仿效节目/文章中的做法吗?\\
    A.从不\\
    B.偶尔\\
    C.经常
    
    \item 
   您能坚持节目/文章里面的做法吗?\\
    A.我能坚持做\\
    B.坚持不久就放弃了
	\end{enumerate}
\item 
您的
性别\underline{\makebox[6em]{}}
年龄\underline{\makebox[6em]{}}
学历\underline{\makebox[6em]{}}
年收入情况\underline{\makebox[6em]{}}
\newline
感谢您的参与以及对我们的支持!
\end{enumerate}

        \subsection{居民访谈问题设计}

\begin{enumerate}
\item 过去的一年您去过几次社区卫生服务中心?

A.0次 \qquad
B.1-2次\qquad
C.2-5次\qquad
D.5次以上

\item 您是因为什么原因去的?(可多选)

A.治疗疾病 \qquad
B.预防保健\qquad
C.职工体检\qquad
D.其他(填写)\underline{\makebox[6em]{}}

\item 您治疗的哪一类疾病?(可多选)

A.上呼吸道感染,如感冒、咳嗽\qquad
B.胃肠疾病\qquad
C.慢性疾病\qquad
D.其他(填写)\underline{\makebox[6em]{}}

\item 您选择了中医治疗还是西医治疗?

A.中医\qquad
B.西医(转至第六题)\qquad
C.都有

\item
您接受了哪一种中医治疗?(可多选)

A.草药煎汤\qquad B.贴敷的膏药\qquad 
C.物理疗法,比如针灸,拔火罐等\qquad D.其他(填写)\underline{\makebox[6em]{}}

\item 您接受了哪一种保健项目?(可多选)

A.草药调理\qquad
B.针灸、推拿\qquad
C.中医体质辨识\qquad
D.贴敷膏药\qquad
E.膏方

\item 您有听说过“治未病”这一概念吗?

A.有\qquad
B.没有\qquad
C.不清楚

\item 您认为“治未病”一词的含义是指什么?(可多选)

A.预防未发生的疾病\qquad B.生病后防止病情进展\qquad C.病后预防疾病再次复发

\item 您了解中医养生知识的途径是以下哪些?(可多选)

A.社区宣传\qquad B.亲戚朋友推荐\qquad C.医生\qquad
D.电视\qquad
E.互联网\qquad
F.其他途径(填写)\underline{\makebox[6em]{}}

\item 您愿意在防病这方面做哪些事情?

A.饮食调理\quad
B.运动调理\quad
C.定期体检\quad
D.主动咨询\quad
E.草药调理\quad
F.物理方法调理,针灸等。

\item 每个月您愿意在防病方面投入多少钱?主要支出如下:

A.一般体检一年花费600左右,每月50块。

B.额外草药支出,每月不定

C.相关课程培训,如太极

D.咨询费用

\end{enumerate}
\end{spacing}
\end{document}