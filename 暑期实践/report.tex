\documentclass{article}
\usepackage{ctex}
\usepackage{hyperref}
\usepackage{setspace}
\usepackage[left=3cm,right=3cm]{geometry}
\author{付裕刚}
\title{治未病暑期实践报告}
\date{\today}
\begin{document}
    \maketitle
    \clearpage
    \section{暑期实践相关思路和目的}
    本次暑期实践的基本目的是做先前设计好的知信行调查问卷,该问卷题量为30题左右,目的是测知居民对于治未病的基本了解和做法。鉴于有一定的题干阅读量,我们建议参与人员年龄不超过65岁以适应其要求的一定知识水平。另外,本文中的社区医院和社区卫生服务中心一词同义。
    
    就整个挑战杯项目而言,我们在挑战杯官方网站的作品库-社会分类下找到往期的作品进行参考。可以发现,近几年的作品要求明显较以往提高,各个获奖作品对于所研究的内容有具体的机制互动设计和阐释,和09年左右泛泛的广泛调查研究区别较大。
    
    同样是南京市的社区研究,来源自第13届挑战杯的作品提出构建“医院-社区-患者-志愿者”四位一体慢性病防治管理新模式。详见下面绿色标出的链接:\href{http://2017.tiaozhanbei.net/d37/project/124/}{慢性病模式构建}。
    
    该项目同时使用了定性和定量两种方法进行研究,定性主要指访谈法,定量主要指问卷法。另外,我预计该团队使用了相关的生存质量量表对于患者的健康状况进行了评估以纠正模式的运行。同时他们吸收了预防医学中“同伴支持”与“志愿服务”两个概念,旨在建立自发性的互助小组和他人干预性的志愿服务相互结合,管理慢性病的发展进程。
    
    对于这一项目,值得借鉴的一点是体系化的思路,可以把平台建立的很完善。通过模式使医院、社区、患者、第三方等利益相关者密切的联系起来,也就可以很方便的把慢性病管理当前现状的优缺点暴露出来并且修正他,让各方之间的协作更加契合。另一点需要注意的是,挑战杯由于需要和相关政策密切结合,因此本着“健康中国2030”的大背景,再细细的梳理“治未病”相关的政策,可以看出预防是这一理念的核心,强调的更多的一种是主动:社区主动宣传,患者主动参与,医院主动服务,继而改变以往“有病才医”的求助式被动局面。但是,走访发现,社区的宣传效力不强,群众对于治未病的认知和意识不够。
    
    那么一种思路是延续使用上述的体系化思路。我们的项目要厘清居民的需求,明晰社区卫生服务中心所能提供的服务范畴和潜力,继而构建一种方式让居民诉求,居委会协调和社区医院的服务形成良性的互动。这样提高三者之间的内在亲和性,提高社区和居民健康之间的联动性。
    
    另一种思路是我们用“冲突”的理论思路,试图去阐述居民和社区医院之间的内在冲突,也就是给出社区医院服务资源浪费和居民健康诉求尚未被满足的原因。继而看“治未病”是否有其特点来缓和这样的冲突。这样的冲突可以是信任/观念问题、也可以是经济成本问题,治未病理论能否扭转观念、降低成本就是值得探讨的问题。
    
    可以看出,无论是采用哪一种思路,都需要先运用访谈法对全体居民进行分层抽样访谈。青年、中年、老年都需要进行访谈以区别他们的需求差异。另外,我们对于社区在治未病工作方面扮演的角色尚不清楚,因此仅访谈社区的居委会主任,并且应该放在对社区卫生服务中心的访谈之后。社区和社区医院之间具体分工和职责的联系可以在访谈之中相互映照,得出结论,不必直接提问以免对方出于谨慎和担忧不愿回答某些问题。对于社区医院的访谈,可以以负责人为线索,再向各个承担治未病服务的科室负责人访谈。
    
    对于定量分析,也就是问卷法而言。理想的方法是先进行抽样,而后在明确住户楼层分布后进行问卷调查。但是在实际操作中,我们发现这样的被拒绝可能性过大,花费的成本太高,因此需要转变方法。对于社区医院的问卷调查针对作为服务方的医生和作为接收方的患者,患者群体较大,应从留有的健康档案中进行抽样以电话回访。如若无法办到则在社区医院随机发放问卷或委托医生交给前来就诊的患者。
    
    \subsection{知信行问卷思路和目的}
    具体见先前的主文件,简单说是对于知识/信念/行为三方面的简单评估。
    
    但是需要注意的是,知信行问卷并未涉及到居民和社区、社区医院之间的互动提问。例如居民去社区医院的频率、对社区医院的满意程度、居民对于社区宣传/组织工作的反馈等等。因为项目成本的问题,另外这和社区医院问卷必定有人群上的重合,因而预计这一信息的获取由居民访谈完成。
        \subsection{卫生服务中心问卷思路和目的}
        对于患者,着重想要了解他们来社区医院的原因、频率、服务项目、建议、对于其他组织参与的认同度等等。
        
        *对于医院,则了解他们对于治未病的看法和当前建设的困难所在。这一工作设计的人群小,也可由访谈完成。
            \subsection{居民访谈思路和目的}
            对于不同年龄层次的居民进行访谈,主要厘清他们对于治未病的印象、期望、生活习惯、投入成本估计、需要解决的健康问题等等
             \subsection{社区访谈思路和目的}
            从居委会了解到居民的来源和年龄分布,以及平日工作中涉及到居民预防保健的部分。我们认为居委会是连结居民和社区医院、第三方服务的纽带之一,因此可以询问居委会是否和卫生服务中心有联动,以及是否知晓第三方志愿者团体的参与等等。
    \section{计划变更}
    暑期问卷发放所需人力不够,改由访谈为主,先进行定性考察。对于问卷发放,先有一小区进行,在人群密集的时间段和地点进行统一服装的问卷发放。另务必登记楼层,以减少同一楼的出现频率过高,造成过大的偏差。
    
    开学后继续招募成员,分组进行走访和问卷的完成,以期望找出现有模式的缺陷,并提出新的具有可行性的某类措施。
    \section{剩余进度}
    从目前的居民访谈来看,居民遇病才会求医,不会主动寻求保健服务,对于保健知识中青年也没有过多的追逐热情。对于老年居民,他们有切身的健康诉求,但是往往由于社区医院的服务不够完善而流向大型三甲医院。
    
    从目前的社区医院访谈来看,以老年人和小孩的就诊为多。其次为职工的定期体检。具体还需要进一步走访社区医院,完善访谈的方法和内容,拿出完整的访谈方案,追溯方法待定。
    
    当前大部分工作尚未完成,同志还需努力。
    \clearpage
    
    \section{相关问卷和访谈设计}
    \subsection{医院访谈设计}
1.您这里有治未病服务吗?具体有哪些?

2.成立了专门的科室吗?还是分散于各个科室?

3.相关科室需要的资金、设备投入?

4.每天的服务量有多少,离得远的社区居民也会来吗?

5.患者经济能力?奔着什么项目来?自费还是医保?

6.响应了哪些相关的治未病政策?

7.服务中心有进行治未病宣传/保健宣传吗?

8.有相关知识宣传、义诊等活动吗?

9.有针对特殊群体,比如针对老年人的服务吗?

11.社区对于治未病有什么规划和期望吗?

13.患者会转去上级医院吗,大致有多少?

14.可访问的相关数据和相应的授权是什么?

目前社区卫生服务的具体实践措施有建立健康档案、妇幼保健、老年人管理、慢性病患者,但是针对年轻群体和健康群体的较少。


    \subsection{居委会访谈设计}
1.居委会联系居民的方式?渠道?频率?

2.居委会在预防保健方面的工作?

3.对于特殊群特的关照?

4.第三方志愿者的组织和活动情况?

5.信息网络化?
    \clearpage
    \begin{spacing}{0.9}
        \subsection{居民访谈问题设计}

\begin{enumerate}
\item 过去的一年您去过几次社区卫生服务中心?

A.0次 \qquad
B.1-2次\qquad
C.2-5次\qquad
D.5次以上

\item 您是因为什么原因去的?(可多选)

A.治疗疾病 \qquad
B.预防保健\qquad
C.职工体检\qquad
D.其他(填写)\underline{\makebox[6em]{}}

\item 您治疗的哪一类疾病?(可多选)

A.上呼吸道感染,如感冒、咳嗽\qquad
B.胃肠疾病\qquad
C.慢性疾病\qquad
D.其他(填写)\underline{\makebox[6em]{}}

\item 您选择了中医治疗还是西医治疗?

A.中医\qquad
B.西医(转至第六题)\qquad
C.都有

\item
您接受了哪一种中医治疗?(可多选)

A.草药煎汤\qquad B.贴敷的膏药\qquad 
C.物理疗法,比如针灸,拔火罐等\qquad D.其他(填写)\underline{\makebox[6em]{}}

\item 您接受了哪一种保健项目?(可多选)

A.草药调理\qquad
B.针灸、推拿\qquad
C.中医体质辨识\qquad
D.贴敷膏药\qquad
E.膏方

\item 您有听说过“治未病”这一概念吗?

A.有\qquad
B.没有\qquad
C.不清楚

\item 您认为“治未病”一词的含义是指什么?(可多选)

A.预防未发生的疾病\qquad B.生病后防止病情进展\qquad C.病后预防疾病再次复发

\item 您了解中医养生知识的途径是以下哪些?(可多选)

A.社区宣传\qquad B.亲戚朋友推荐\qquad C.医生\qquad
D.电视\qquad
E.互联网\qquad
F.其他途径(填写)\underline{\makebox[6em]{}}

\item 您愿意在防病这方面做哪些事情?

A.饮食调理\quad
B.运动调理\quad
C.定期体检\quad
D.主动咨询\quad
E.草药调理\quad
F.物理方法调理,针灸等。

\item 每个月您愿意在防病方面投入多少钱?主要支出如下:

A.一般体检一年花费600左右,每月50块。

B.额外草药支出,每月不定

C.相关课程培训,如太极

D.咨询费用

\end{enumerate}
\end{spacing}
\end{document}