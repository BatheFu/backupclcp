\section{知信行理论模式简介及其应用}
\subsection{简介}
人们已经知道,在知识、信念和行为之间存在着一条沟壑:例如吸烟者知道吸烟存在的健康风险,即使他们了解后秉持了这一信念,他们却很难戒除吸烟的行为。在治未病理论的传播过程中,我们希望受众不仅仅是习得普遍的客体的知识,还能对此形成内化的信念并且真正地把这些知识应用于指导自身的生活。

在众多的理论模型中,知信行理论模式是应用最为广泛的模型,阐释了人的知识增长可以促进信念的加深进而转变为人的行为。知信行理论模式是认知理论和动机理论等在教育中的应用,主要探究了知识、信念和行为三者之前的联系。其中,知识是基础,信念(态度)是动力,行为则是目标。\cite{黄敬亨2006健康教育学}

在知信行模式中,“知”是某些明确的、得到普遍承认的客体知识。“知”代表通过某些媒介来传播知识,从而让人们“认知”和分析对象。“信”是在对知识的客体认知的基础上建立起来的信念,对于本研究而言,即是指受众在接受相关治未病知识后,树立起的对于治未病理论的信念和对对应行为的愿意程度。“行”是这一模式的行为层次,指前述的知识和信念已经部分或全部的转变成实际行动。\cite{金新政2003}

\subsection{知信行模式在医学领域的应用}基于此模式进行的对照组研究在护理医学、预防医学领域形成了一套相对完备的体系,广泛应用于糖尿病、高血压、呼吸系统慢性疾病的患者院外自理,术后康复,常见疾病预防等等。

通过知信行模式,患者的院外自理能力加强,生活质量得到了提高,有的患者通过这一教育模式还形成了新的适应自身特点的生活模式。但是,知信行模式在和中医相关的治未病理论传播和实践当中的应用尚且缺乏,这也为我们的研究提供了创新的前景和空间。

\subsection{“治未病理论”和知信行模式}
为了阐述治未病理论和知信行模式的对应关系,需要确定“治未病理论”中的知识、信念以及行为意味着什么。这一个部分也为后面的问卷条目的编写提供了理论支持。
\subsubsection{客体知识的界定}
由于中医学科的体系和现代科学有差异,中医知识范畴的确定不是使用概念和子概念这一体系来确定的,而是在不断的实践当中逐渐生成,在文本之中的对比、比较之后体会得到。对于需要界定为“客体知识”的名词,尽量使用一些公认的说法,此类说法往往出现在教科书上,来避免分歧和不必要的讨论。此外,治未病的客体知识主要从两个部分进行确定。

一是从中医基础理论中确定的古人对于人体、疾病、环境、世界的认识和其运行、发生的机制解释。涉及的名词包括阴阳、五脏关系、气血津液、四时五行等等。

二是从长期的治未病实践中寻找广泛有效的做法,这些做法可以看作是治未病理论在实际中的应用手段。涉及的名词包括气功、太极拳、五禽戏、精神调养等等。

\subsection{信念的界定}
对于治未病理论体系的信念可以分为“抽象信念”和“行为信念”两部分。其中,“抽象信念”指的是人们对于知信行体系本身的内容以及他可能联想到的中医印象以及治未病理论中折射出传统中国文化的相关内容的信念和认可;“行为信念”指的是人们对于自身或周围的人参与治未病实践活动、接受相应的咨询/治疗的愿意程度。
\subsection{行为的界定}
行为的界定较为容易和明显,指的是人们主动了解治未病体系的知识,或者主动通过多种媒介获取治未病相关信息;切实参与进治未病实践活动、接受相应的咨询/治疗。



