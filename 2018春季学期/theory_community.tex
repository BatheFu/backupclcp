\section{当前社区健康体系建设}
\subsection{健康管理}
健康管理是对个体或群体的健康进行全面监测、分析、评估、提供健康咨询和指导以及对健康危险因素进行干预的全过程。健康管理的宗旨就是充分调动个体和群体以及整个社会的积极性,有效利用有限的资源来达到最大的效果。

1929年,美国蓝十字和蓝盾保险公司进行疾病管理实践与探索中首次提出健康管理理念。1969年,美国政府将健康维护组织纳入国家医疗保障体系中,并于1971年为其立法,逐步形成了一个系统化管理的健康观。上世纪末,健康管理的理念开始引进我国。

在2008年卫生部全国卫生工作会议上正式提出实施“健康中国2020”战略,其目标是到2020年,建立覆盖城乡居民的比较完善的基本医疗卫生制度,缩小经济、社会发展水平差异造成的健康不平等现象,实现“人人享有基本医疗卫生服务目标,促进卫生服务利用的均等化,大幅提高全民健康水平”。

党的十八届五中全会提出了“推进健康中国建设”的发展战略,特别突出以人的健康为中心。我国的医疗卫生事业的改革方向已从解决人民群众看病就医问题向促进和保障人民健康转变。

\subsection{社区卫生服务}
社区卫生服务是社区服务中的一种最基本、普遍的服务,是由全科医生为主要卫生人力的卫生组织或机构所从事的一种社区定向的卫生服务,与医院定向的专科服务有所不同,它是社区建设的重要组成部分,以人的健康为中心、以家庭为单位、社区为范围、需求为导向,以妇女、儿童、老年人、慢性病患者、残疾人、低收入居民为重点,以解决社区主要卫生为题,满足基本医疗卫生服务需求为目的,融预防、医疗、保健、康复、健康教育和计划生育技术服务等为一体的,有效的、经济的、方便的、综合的、连续的基层卫生服务。

从1985年的医疗卫生体制改革以来,“看病难、看病贵”的问题一直困扰着我国的医疗卫生体制。与西方成熟的社区医疗卫生模式相比,我国社区医疗起步较晚,于1996年开始试点,全面推动社区卫生服务体系的建设则在2000年开始进行。我国城市社区卫生服务的多年实践经验与国外社区卫生服务发展的成功经验告诉我们,发展社区卫生服务可以有效地解决居民“看病难、看病贵”的问题。

一方面,发展社区卫生服务可以分流大医院的患者,解决居民“看病难”的问题,另一方面,社区卫生服务机构的预防保健、健康教育等服务可以有效提高居民自我保健意识,降低疾病发病率,同时,社区卫生服务机构提供的基本医疗服务价格比大医院低的多,可以在一定程度上降低居民的医疗支出,解决“看病贵”的问题。

\subsubsection{我国社区卫生服务体系}
社区卫生服务体系主要是在城镇居民中设立社区卫生服务中心,再根据其社区覆盖面积及人口,在中心下设若干社区卫生服务站,以利于附近居民就诊和接受健康教育、康复及照顾等。社区卫生服务人员主要由全科医师、护士及专业卫生技术和管理人员组成。

我国有些城市将一级医院转为社区卫生服务中心,或将原二、三级医院与新设的社区卫生服务中心加强联系,以利指导,实施双向转诊,使各项基本卫生服务逐步得到有机融合,并形成基层卫生服务网络,提高基层卫生服务的水平与质量。一些街道办事处在完善社区功能时,亦设社区卫生服务中心,与附近基层医院共同建设及培训人员。

\subsubsection{社区卫生服务进展情况}
根据卫生部统计信息中心数据统计显示,2002年我国共有社区卫生服务中心(站)8211家,发展至今已经覆盖了全国三十多个省、自治区、直辖市。社区卫生服务机构数量逐年攀升,其中,2008年社区卫生服务机构数量减少的原因是江苏省5000家农村社区卫生服务站划归村卫生室。

为加强社区卫生服务机构内涵建设,卫生部于2011年1月启动了创建示范社区卫生服务中心活动,确定北京市西城区展览路社区卫生服务中心等141个机构,2012年确定北京市朝阳区劲松社区卫生服务中心等164个机构,2013年确定朝阳区亚运村社区卫生服务中心等183个机构为全国示范社区卫生服务中心……力求进一步在完善服务功能、创新服务模式的同时为社区居民提供方便、经济、安全、有效的基本医疗和公共卫生服务。

\subsection{治未病理论体系下社区预防工作的开展}
社区卫生服务中心应向社区公众及单位提供全方位的预防保健服务,治未病理论体系下的预防工作内容应包括以下几方面:
\begin{itemize}
\item 开展慢性非传染性疾病的预防工作。重点抓好高血压、冠心病、脑卒中、糖尿病、肿瘤等疾病的预防,开展疾病监测及危险因素调查,提供针对个体的预防措施。
\item 开展营养指导。重点抓好孕妇、乳母、少儿、老年及接触职业性有害因素人群的营养保健指导,指导居民科学选择保健食品。
\item 指导社区居民搞好个人卫生、居室卫生、环境卫生,做好各项卫生防护,预防家用化学制品、化妆品、建筑装饰材料及生活用品对健康产生不良影响。
\item 开展经常性的健康教育,提供预防保健心理咨询服务与卫生防病技术服务,定期向社区居民通报疫情动态,提供社区居民防病保健意识,普及自我保健知识。
\item 指导社区居民搞好饮食供水卫生,预防食源性疾患、食物中毒及水传性疾病。
\item 开展预防性健康检查,建立健康档案。

按照杜区卫生服务六大功能要求,应突出预防保健和健康教育,然而实际情况并非如此,社区卫生服务的主体,社区卫生服务中心及派出机构—卫生服务站是在基层医院的基础上重组成立的,从事医疗和康复工作的医生、护士占绝大多数气由于受传统服务模式的影响,在实际工作中还存在重治轻防的倾向,防保人员数量不足,预防项目开展的少,综合性预防工作还不到位。

社区服务是为适应新的医学模式而建立起来的新的卫生机构,因此不能沿用传统服务模式,经从组织形式、服务内容和方式等方面应有所创新,特别是社区卫生服务站不能办成“看病拿药”的医疗服务站,应把预防工作放在首位,在社区服务中突出预防工作,只有这样才能有利于全面贯彻预防为主的方针,有利于充分利用城市卫生资源加强预防工作,把预防工作落实到千家万户,使公众得到更多的预防保健方面的实惠。
\end{itemize}





