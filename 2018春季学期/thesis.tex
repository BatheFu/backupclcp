\documentclass{article}
\usepackage{ctex}
\usepackage{graphicx}
\usepackage{longtable}
\usepackage{multicol}
\usepackage{multirow}
\usepackage[left=3cm,right=3cm]{geometry}
\graphicspath{{images/}}
\usepackage[backend=biber,style=gb7714-2015]{biblatex}
\addbibresource[location=local]{refer.bib}	
\title{南京市社区治未病“知信行”现状调查}
\author{付裕刚}
\date{\today}
\begin{document}
    \maketitle
\section{对象和方法}
\subsection{对象}
选取南京市五个社区,分属不同的五个区划,每个社区的人数在6-8千人。统计各社区楼栋数,根据居委会的提示,排除青年人或老年人占绝大多数的楼层,随机选取居民进行自填式问卷调查。共调查192户,回收问卷265份。其中男性85人,女性180人,比例为32:68。
\subsection{方法}
\subsubsection{问卷设计}
根据先前进行的访谈,以及可进行参考的文献\cite{cjw_1_2009}\cite{wcy2011}\cite{kam2005}并咨询专家自行设计问卷,包括治未病知识、信念和行为三个维度。

其中,知识维度包括10条单选和2条多选。其中第1题为筛选题,不计分。
第2题承接第1题的引导,为多选,考察治未病的含义,每个选项计1分,共三分。
第3-10题采用4级评分法(正确=2分,部分正确=1分,错误=0分,不清楚=NA)。共计19分。
第11题为多选题,了解了知识获取的途径,不计分。故知识维度共19分。

信念维度共8条单选,序号为12-19,使用4级评分法(相信/愿意/喜欢=2分,部分相信/部分愿意/部分喜欢=1分,不相信/不愿意/不喜欢=-2分、不好说=NA)。共计16分。

行为维度共6题。序号为20-26,其中第20题设置跳题逻辑,选择(A=0次)跳过21题,分值为0、1、2、3。21为多选,考察治疗手段。22-26为单选,同样采用差额计分,最高项2分。其中26有4小题。共计24分。

具体可见下表
\newpage

\begin{longtable}[]{@{}llll@{}}

题号 & 维度 & 题目类型 & 是否有跳题\tabularnewline
\hline
\endhead
1 & 知识 & 单选 & 是\tabularnewline
2 & 知识 & 多选 & 否\tabularnewline
3-10 & 知识 & 单选 & 否\tabularnewline
11 & 知识 & 多选 & 否\tabularnewline
12-19 & 信念 & 单选 & 否\tabularnewline
20 & 行为 & 单选 & 是\tabularnewline
21 & 行为 & 多选 & 否\tabularnewline
22-26 & 行为 & 单选 & 否\tabularnewline
\hline
\caption{表中各题维度、类型、跳题设置}
\end{longtable}

此外还有匿名个人信息统计条目:

\begin{longtable}[]{@{}lll@{}}

题号 & 变量名称 & 变量类型\tabularnewline
\hline
\endhead
27 & 性别 & 分类\tabularnewline
28 & 年龄 & 数值\tabularnewline
29 & 学历 & 分类\tabularnewline
30 & 年收入 & 分类\tabularnewline
\hline
\end{longtable}


\subsubsection{统计学方法}
使用R语言作为工具,计数资料采用例数、百分比进行描述;计量资料采用均数、标准差进行描述,分类变量采用 $\tau$ 检验和单因素方差分析。

\section{结果}

\section{人口学特征}
265例中男性85人,女性180人。年龄区间<20岁的32人,20-30岁82人。30-40岁48人,40-50岁73人,50岁以上28人。
教育水平,小学学历2人,初中学历19人,高中学历50人,大学本科学历161人,研究生及以上学历33人。
年收入水平5万以下143人,5-10万72人,10-30万43人,30-50万4人,50万以上3人。

\section{得分特征}
采用均值-标准差分析,得出得分区间和得分率。

%知识维度
第3-10题,以各列均值替代缺失值,得出得分区间为$\hat{x}\pm s = 14.47 \pm 3.88$ ,平均得分率为 $0.761$。

%信念维度
第12-19题,以各列均值替代缺失值,得出得分区间为$\hat{x}\pm s = 12.10 \pm 3.48$ ,平均得分率为 $0.756$。

%信念维度
第12-19题,以各列均值替代缺失值,得出得分区间为$\hat{x}\pm s = 9.56 \pm 2.49$ ,平均得分率为 $0.398$。

分组检验的结果见excel表。

\section{讨论}
\begin{enumerate}
    \item 近半数受访者表示没听说“治未病”一词。
    
    尽管学界关于“治未病”理论的探讨已经有近六十年的历史,但是根据这一样本,这一概念的普及率仍只有46\%。而对于听说过这一概念的受访者,对于“治未病”具体含义的理解,有64\%选择了“预防未发生疾病”、15\%选择了“生病后防止病情进展”、22\%选择了“病后预防疾病再次复发”。、
	
	将听说过“知信行”概念和没有听说过“知信行”分为两组,$H_0$假设为二者知信行得分无差异,t检验后发现P>0.05,无显著性差异,再通过二者的均值可以看出,知识维度高2.738分,信念维度高0.997,行为维度高2.683。可见听说过“治未病”一词的群体对知信行理论更加了解,态度更积极,行为参与度高。
	
    \item 养生信息获取渠道分析
    第11题询问了受调查者获取养生知识的途径。其中79人选择社区宣传、155人选择亲戚朋友推荐、145人选择医生、155人选择电视、176人选择互联网、24人选择其他途径。可以看出,社区宣传在其中占比较小,仅占约30\%。
	建议社区开展主动式的养生知识普及,向有实际需求但是获知信息能力弱的老年人介绍节气养生,食疗、养生按摩等基础知识。

	
	\item 知识、信念得分和行为得分的反差对比
	
	通过Pearson相关系数分析得到系数矩阵,知识-信念、知识-行为、信念-行为之间的相关系数分别为0.539,0.304,0.208,P均小于0.001,说明三者之间为正相关。
	得分率上,知识维度和维度得分率都较为满意,百分比分别为76.1和75.6,但是行为维度陡降至39.8,可见“信而不行”的情况。分析可能的原因有以下几点。
	一是受调查者身体健康,没有就医或者保健的行动欲望,因而尽管对治未病知识有一定了解,并且对中医治未病理念有信心,但是不会做出实际行动。二是在问卷前期访谈中,部分受调查者认为平常获取的养生信息可信度不高,半信半疑之间,不会付诸实际行动。
	
\end{enumerate}


\printbibliography
\appendix{}
\section{问卷小样}
您好,这是一份由南京中医药大学第一临床医学院发起的调查问卷,旨在了解社区居民对于中医治未病理论的认知、行为等情况,响应“健康中国”工程,为后面中医治未病理论的传播做一个描述性统计分析。

本问卷时长约五分钟,我们郑重承诺,本问卷的所有数据仅用于科研用途,您的所有个人信息将受到相关法律的保护。

\begin{enumerate}

\item
偏方是中医的一种。\\
A.正确\\
B.部分正确\\
C.错误

\item
煎药时间越久越好。\\
A.正确\\
B.部分正确\\
C.错误

\item
食疗是中医的治疗手段\\
A.正确\\
B.部分正确\\
C.错误

\item 
治未病是中医特有的理论和方法
A.正确\\
B.部分正确\\
C.错误

\item 中医的存亡取决于中医是否有效
A.正确\\
B.部分正确\\
C.错误

\item
过去的一年里,您接受过中医治疗吗?(Hon第一题)\\
A. 有\\
B. 没有

\item
过去一年您接受中医治疗的次数?\\
A.1-2次\\
B.2-5次\\
C.5次以上

\item
您接受了哪一种中医治疗呢?\\
A.草药煎汤\\
B.贴敷的膏药\\
C.物理疗法,比如针灸,拔火罐等\\
D.填写

\item 
您参加过养生知识的讲座吗?
A. 有\\
B. 没有

\item 
未来您会去接受中医治疗吗?\\
A.会\\
B.不会\\
C.可能


\item 您喜欢看养生方面的书籍、视频等资料吗?
A.喜欢\\
B.一般\\
C.不喜欢\\

\item 
您对于当前国家鼓励中医药发展怎么看?\\
A.我支持\\
B.我保持中立\\
C.我反对

\item 
我们知道,电视台会播放中医节目,也有微信文章进行中医知识的传播。
\begin{enumerate}
	\item 
   您收听/观看此类节目吗?\\
    A.从不\\
    B.偶尔\\
    C.经常

\item 
    您阅读此类文章吗?\\
    A.从不\\
    B.偶尔\\
    C.经常
    
    \item 
    您喜欢这些节目/文章吗?\\
    A.喜欢\\
    B.一般\\
    C.不喜欢
    
    \item 
    您会仿效节目/文章中的做法吗?\\
    A.从不\\
    B.偶尔\\
    C.经常
    
    \item 
   您能坚持节目/文章里面的做法吗?\\
    A.我能坚持做\\
    B.坚持不久就放弃了
	\end{enumerate}
\item 
您的
性别\underline{\makebox[6em]{}}
年龄\underline{\makebox[6em]{}}
学历\underline{\makebox[6em]{}}
年收入情况\underline{\makebox[6em]{}}
\newline
感谢您的参与以及对我们的支持!
\end{enumerate}

\end{document}